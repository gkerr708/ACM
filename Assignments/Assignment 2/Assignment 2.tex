\documentclass[12pt, a4paper]{article}
\usepackage{amsfonts, amssymb,amsmath}
\usepackage{graphicx}
\usepackage{dsfont}
\usepackage[margin=1.0in]{geometry}
\DeclareMathOperator\arctanh{arctanh}
\linespread{1} 

\title{PHYC 3590 - Advanced Classical Mechanics\\Assignment 2}
\author{Gavin Kerr\\B00801584}
\date{2022-01-24}


\begin{document}

\maketitle


\section*{Question 2.10}
\subsection*{(a)}
These variables are given in the question.
\begin{align*}
\eta =& 12 \, N\cdot s/m^2
\\
\rho =& 7.8 \, g/cm^3
\\
D =& 2 \, mm
\end{align*}
The forces acting on the ball are 
\begin{align*}
\vec{F}_{lin} =& - 3\pi \eta D v\hat{y}
\\
\vec{F}_{buoyant} =& -\rho g V \hat{y}
\\
\vec{F}_{g} =& mg\hat{y}
\end{align*}
where the ball is falling in the $+y$ direction. For the net force we have
\begin{align*}
ma =& mg - 3 \pi \eta D v - \rho g V   
\end{align*}
For $v_{ter}$ we set the $a$ to equal $0$.
\begin{align}
0 =& mg - 3 \pi \eta D v_{ter} - \rho g V  
\\
3 \pi \eta D v_{ter} =& mg - \rho gV 
\\
v_{ter} =& \frac{mg - \rho gV}{3 \pi \eta D}
\\
v_{ter} =& \frac{mg - \rho gV}{b} \label{eqn idk}
\end{align}
\begin{align}
ma =& mg - b v - \rho g V   
\\
\frac{m}{b}a =& \frac{mg - \rho gV}{b}   -  v \label{eqn 3}
\end{align}
Subbing equation (\ref{eqn idk}) into equation (\ref{eqn 3}) gives
\begin{align*}
\frac{m}{b}a =& v_{ter} -  v
\end{align*}
Let $u = v - v_{ter}$
\begin{align*}
m\dot{u} =& -bu
\\
u =& u_0 \exp \left( -\frac{t}{\tau} \right), \,\, \tau = \frac{m}{b}
\\
v - v_{ter} =& (v_0 - v_{ter})\exp \left( -\frac{t}{\tau} \right)
\end{align*}
\begin{align*}
\tau =& \frac{m}{b}
\\
\tau =& \frac{m}{3\pi\eta D}
\\
m =& \rho V = \rho \frac{4}{3}\pi (D/2)^3 = \frac{1}{6}\rho\pi D^3
\\
\tau =& \frac{\rho\pi D^3}{6 \cdot 3\pi\eta D} = \boxed{\frac{\rho D^2}{18\eta}}
\\
\tau =& \frac{(7.8g/cm^3) (2mm)^2}{18(12N\cdot s/m^2)}
\\
\tau =& \frac{(7800kg/m^3) (0.002m)^2}{18(12N\cdot s/m^2)}
\\
\tau =& \boxed{1.4\times10^{-4} \, seconds}
\end{align*}
95\% of $v_{ter}$ is reached after $3\tau$, which is equal to $\boxed{4.3\times10^{-4} \, seconds}$. \\
\\
Solving for $v_{ter}$ numerically, we have
\begin{align*}
v_{ter} =& \frac{mg - \rho_{gly} gV}{b}
\\
v_{ter} =& \frac{mg}{3\pi\eta D} - \frac{\rho gV}{3\pi\eta D}
\\
v_{ter} =& \frac{\tfrac{1}{6}\rho_{ball}\pi D^3 g}{3\pi\eta D} - \frac{\rho_{gly} g(\tfrac{1}{6}\pi D^3)}{3\pi\eta D}
\\
v_{ter} =& 
\frac{\tfrac{1}{6}\rho_{ball} D^2 g}{3\eta } - 
\frac{\tfrac{1}{6}\rho_{gly } D^2 g}{3\eta }
\\
v_{ter} =& 
\frac{D^2 g}{18\eta } (\rho_{ball}-\rho_{gly }) 
\\
v_{ter} =& 
\frac{(0.002m)^2 (9.81m/s^2)}{18(12N\cdot s/m^2) } 
(7800kg/m^3 - 1300kg/m^3) = \boxed{1.2\times10^{-3}m/s}
\end{align*}
\subsection*{(b)}
\begin{align*}
F_{lin} =& 3\pi \eta D v
\\
R =& \frac{Dv\rho}{\eta}
\\
\frac{F_{quad}}{F_{lin}} =& \dfrac{R}{48}
\\
\frac{F_{quad}}{F_{lin}} =& \frac{Dv\rho}{48\eta}
\\
\frac{F_{quad}}{F_{lin}} =& \frac{(0.002mm)(1.2\times10^{-3}m/s)(1.3g/cm^3)}{48(12N\cdot s/m^2)}
\\
\frac{F_{quad}}{F_{lin}} =& \frac{(0.002m)(1.2\times10^{-3}m/s)(1300kg/m^3)}{48(12N\cdot s/m^2)}
\\
\frac{F_{quad}}{F_{lin}} =& \boxed{5.4\times10^{-6}}
\end{align*}
$\frac{F_{quad}}{F_{lin}}$ is small, therefore $F_{quad}$ is negligible. This means it was a good approximation to only consider linear drag. 





\pagebreak
\section*{Question 2.19}
\subsection*{(a)}
The only force acting on the projectile is $\vec{w} = -mg\hat{y}$ therefore $a = -g\hat{y}$. The particle also has an initial velocity, $\vec{v}_0 = v_{x_0}\hat{x}+v_{y_0}\hat{y}$. Next to solve for the equation for motion we have
\begin{align}
x =& \int v_{x_0}\, dx = \boxed{v_{x_0}t} \label{x eqn}
\\
y =& \int v_{y_0}\, dx - \int\int g \,\, dx \, dx = \boxed{
v_{y_0}t - \tfrac{1}{2}gt^2} \label{y eqn}
\end{align}
Next we use equation (\ref{x eqn}) to solve for $t$, then plug it into question (\ref{y eqn}). 
\begin{align*}
t =& \frac{x}{v_{x_0}}
\\
y =& v_{y_0}\left(\frac{x}{v_{x_0}}\right) - \tfrac{1}{2}g\left(\frac{x}{v_{x_0}}\right)^2
\\
y =& \boxed{\frac{v_{y_0}}{v_{x_0}}x - \frac{g}{2v_{x_0}^2}x^2}
\end{align*}
\pagebreak
\subsection*{(b)}
Equation 2.37 from Taylor is 
\begin{align*}
y =& \frac{v_{y_0}+v_{ter}}{v_{x_0}}x + v_{ter}\tau\ln\left( 1 - \frac{x}{v_{x_0}\tau} \right)
\end{align*}
When the air resistance is switched off, $v_{ter}$ and $\tau$ go to $\infty$.
\begin{align*}
\lim_{v_{ter}, \tau \rightarrow\infty} (y) =& ???
\end{align*}
Taylor expanding the $ln$ term from the equation for $y$ gives
\begin{align*}
y =& \frac{v_{y_0}+v_{ter}}{v_{x_0}}x - v_{ter}\tau \left[
\frac{x}{v_{x_0}\tau} + 
\frac{1}{2}\left( \frac{x}{v_{x_0}\tau} \right)^2 +
\frac{1}{2}\left( \frac{x}{v_{x_0}\tau} \right)^3 + \dots
\right]
\\
y =& \frac{v_{y_0}+v_{ter}}{v_{x_0}}x - v_{ter}\left[
\frac{x}{v_{x_0}} + 
\frac{1}{2\tau}\left( \frac{x}{v_{x_0}} \right)^2 +
\frac{1}{2\tau^2}\left( \frac{x}{v_{x_0}} \right)^3 + \dots
\right]
\\
y =& \frac{v_{y_0}}{v_{x_0}}x + \frac{g\tau}{v_{x_0}}x - g\tau\left[
\frac{x}{v_{x_0}} + 
\frac{1}{2\tau}\left( \frac{x}{v_{x_0}} \right)^2 +
\frac{1}{2\tau^2}\left( \frac{x}{v_{x_0}} \right)^3 + \dots
\right]
\\
y =& \frac{v_{y_0}}{v_{x_0}}x + \frac{g\tau}{v_{x_0}}x - g\tau
\frac{x}{v_{x_0}} - 
g\tau \left[
\frac{1}{2\tau}\left( \frac{x}{v_{x_0}} \right)^2 +
\frac{1}{2\tau^2}\left( \frac{x}{v_{x_0}} \right)^3 + \dots
\right]
\\
y =& \frac{v_{y_0}}{v_{x_0}}x - 
g\tau \left[
\frac{1}{2\tau}\left( \frac{x}{v_{x_0}} \right)^2 +
\frac{1}{2\tau^2}\left( \frac{x}{v_{x_0}} \right)^3 + \dots
\right]
\\
y =& \frac{v_{y_0}}{v_{x_0}}x - 
g \left[
\frac{1}{2}\left( \frac{x}{v_{x_0}} \right)^2 +
\frac{1}{2\tau}\left( \frac{x}{v_{x_0}} \right)^3 + \dots
\right]
\end{align*} 
\begin{align*}
\lim_{\tau\rightarrow\infty}(y) =& \boxed{\frac{v_{y_0}}{v_{x_0}}x - 
\frac{g}{2v_{x_0}^2}x^2 }
\end{align*}
\begin{align*}
\text{QED}
\end{align*}





\pagebreak
\section*{Question 2.37}
The first part of this problem is to solve this integral 
\begin{align}
&\int \frac{1}{1-u^2} \, du
\\
=& \frac{1}{2}\left[ \int \frac{1}{1+u} du +  \int \frac{1}{1-u} du \right] \label{integral}
\end{align}
For the left most term we let $v = 1+u$, $du = dv$
\begin{align*}
\int \frac{1}{1+u} du  =& \int \frac{1}{v} dv
\\
=& \ln v
\\
=& \boxed{\ln (1+u)}
\end{align*}
For the right most term we let $v = 1-u$, $du = -dv$
\begin{align*}
\int \frac{1}{1-u} du =& -\int \frac{1}{v} dv
\\
=& -\ln v
\\
=& \boxed{-\ln (1-u)}
\end{align*}
Subbing the boxed equations int equation (\ref{integral}) gives 
\begin{align}
&\frac{1}{2}\left[ \ln(1+u)  - \ln(1-u) \right]
\\
=& \frac{1}{2}\ln\left[\frac{1+u}{1-u}\right]
\\
\int \frac{1}{1-u} du=& \arctanh u \label{u equation}
\end{align}
From the book we have 
\begin{align*}
\dfrac{dv}{1-v^2/v_{ter}^2} =& g \, dt
\\
\int_0^v\dfrac{dv}{1-v^2/v_{ter}^2} =&  \int_0^t g \, dt
\\
v_{ter}\int_0^{u}\dfrac{du'}{1-u'^2} =&  \int_0^t g \, dt
\\
u = v/v_{ter}, &\,\, dv = v_{ter}dt
\end{align*}
Using the solution given by equation (\ref{u equation}), we find 
\begin{align*}
v_{ter} \arctanh \frac{v}{v_{ter}} =&  gt
\\
\arctanh \frac{v}{v_{ter}} =&  \frac{gt}{v_{ter}}
\\
\frac{v}{v_{ter}} =&  \tanh\left(\frac{gt}{v_{ter}}\right)
\\
v =&  \boxed{v_{ter}\tanh\left(\frac{gt}{v_{ter}}\right)}
\\
\text{QED}&
\end{align*}







\pagebreak
\section*{Question 2.52}
\begin{align*}
\eta =& A e^{-i\omega t} 
= ae^{i\delta}e^{-i\omega t}
= ae^{i(\delta-\omega t)}
\\
\eta =& a\cos(\delta-\omega t)+ia\sin(\delta-\omega t)
\\
v_x =& a\cos(\delta-\omega t), \,\, v_y = a\sin(\delta-\omega t)
\end{align*}
$\eta$ moves clockwise with a constant speed of $|a|$ and an angular velocity of $-\omega$. 








\section*{Question 2.54}
Equations (2.68) are
\begin{align*}
\dot{v_x} =& \omega v_y
\\
\dot{v_y} =& -\omega v_x
\end{align*}
Differentiating the first equation in (2.68), we have
\begin{align*}
\ddot{v_x} =& \omega \dot{v_y}
\\
\frac{\ddot{v_x}}{\omega} =&  \dot{v_y}
\end{align*}
Subbing this result into the second equation in (2.68) gives
\begin{align*}
\frac{\ddot{v_x}}{\omega} =& -\omega v_x
\\
\ddot{v_x} =& -\omega^2 v_x
\\
v_x =& \boxed{A\sin(\omega t) + B\cos(\omega t)}
\end{align*}
Now solving for $v_y$ we have
\begin{align*}
\omega v_y =& \dot{v_x} 
\\
\omega v_y =& \frac{d}{dt}[A\sin(\omega t) + B\cos(\omega t)]
\\
\omega v_y =& A\omega\cos(\omega t) - B\omega\sin(\omega t)
\\
v_y =& \boxed{A\cos(\omega t) - B\sin(\omega t)}
\end{align*}
\begin{align*}
\vec{v} =& A\sin(\omega t) + B\cos(\omega t) + i(A\cos(\omega t) - B\sin(\omega t))
\end{align*}

































\end{document}
