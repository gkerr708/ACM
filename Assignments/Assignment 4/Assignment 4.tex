\documentclass[12pt, a4paper]{article}
\usepackage{amsfonts, amssymb,amsmath}
\usepackage{graphicx}
\usepackage{dsfont}
\usepackage[margin=1.0in]{geometry}
\linespread{1} 

\title{PHYC 3590 - Advanced Classical Mechanics\\Assignment 4}
\author{Gavin Kerr\\B00801584}
\date{2023-02-07}


\begin{document}

\maketitle









\section*{Problem 4.2}
\subsection*{(a)}
\begin{align*}
W =& \int_0^P F \cdot dr
\\
=& \int_0^Q F \cdot dr + \int_Q^P F \cdot dr
\\
=& \int_0^1 F_xdx + \int_Q^P F_ydy
\\
=& \int_0^1 x^2 dx + \int_Q^P 2 xy \,dy
\\
=& \left[ \frac{x^3}{3} \right]_0^1 + \int_Q^P 2 xy \,dy
\\
=& \frac{1}{3}  + \int_Q^P 2 xy \,dy
\end{align*}
On the path from Q to P, x is equal to 1. 
\begin{align*}
=& \frac{1}{3}  + \int_0^1 2y \,dy
\\
=& \frac{1}{3}  + [y^2]_0^1
\\
W =& \frac{1}{3}  + 1 
\\
W =& \boxed{\frac{4}{3}}
\end{align*}
\subsection*{(b)}
\begin{align*}
W =& \int_0^P F \cdot dr
\\
=& \int_0^P F_x dx + \int_0^P F_y dy
\\
=& \int_0^P x^2 \, dx + \int_0^P 2xy \, dy
\\
y =& x^2, \,\,\, dy = 2x \,dx
\\
W=& \int_0^P x^2 \, dx + \int_0^P 4x^4 \, dx
\\
=& \int_0^1 x^2 \, dx + \int_0^1 4x^4 \, dx
\\
=& \left[ \frac{x^3}{3} \right]_0^1 + 
   \left[ \frac{4x^5}{5}\right]_0^1
\\
=& \frac{1}{3} + \frac{4}{5} 
\\
W =& \boxed{\frac{17}{15}}
\end{align*}
\subsection*{(c)}
\begin{align*}
W =& \int_0^{x=1} x^2 \, dx + \int_0^{y=1} 2xy \, dy
\\
W =& \int_0^{t=1} t^6 \, dx + \int_0^{t=1} 2t^3t^2 \, dy
\\
dx =& 3t^2 \,dt, \,\,\, dy = 2t \,dt
\\
W =& \int_0^{t=1} t^6 \cdot 3t^2 \,dt + \int_0^{t=1} 2t^5 \cdot 2t \,dt
\\
W =& \int_0^1 3t^8 \,dt + \int_0^1 4t^6 \,dt
\\
W =& \left[ \frac{2x^9}{9} \right]_0^1 + 
   \left[ \frac{4x^7}{7}\right]_0^1
\\
W =& \frac{3}{9} + \frac{4}{7}
\\
W =& \boxed{\frac{19}{21}}
\end{align*}







\pagebreak
\section*{Problem 4.8}
\begin{align*}
\vec{F} =& \vec{W} + \vec{N}
\end{align*}
In the $\hat{x}$ vector we have:
\begin{align*}
F_x =&  mg\sin\theta
\end{align*}
In the $\hat{y}$ vector we have:
\begin{align*}
F_y =& mg + mg\cos\theta
\end{align*}
For the total energy of the puck we have
\begin{align*}
E =& mgR
\\
E =& T + U
\\
U =& mgh = mgR\cos\theta
\\
T =& \tfrac{1}{2}mv^2
\\
mgR =& \tfrac{1}{2}mv^2 - mgR\cos\theta 
\\
\end{align*}









\pagebreak
\section*{Problem 4.22}
The goad of this problem is to prove that the Couolomp force  is conservative.
\begin{align*}
\vec{F} =& \frac{\gamma}{r^2}\hat{r}
\\
\Delta \times F  =& 
\frac{1}{r\sin\theta}\left[\frac{\partial}{\partial\theta}(\sin\theta F_\phi)-\frac{\partial}{\partial\phi}F_\phi\right]\hat{r} 
+\left[\frac{1}{r\sin\theta}\frac{\partial}{\partial\phi}F_r-\frac{1}{r}\frac{\partial}{\partial r}(rF_\phi)\right]\hat{\theta} 
\\
+& \frac{1}{r}\left[\frac{\partial}{\partial r}(rF_\theta)-\frac{\partial}{\partial\theta}F_r\right]\hat{\phi}
\end{align*}
$\vec{F}$ is equal zero in the $\hat{\theta}$ and $\hat{\phi}$ direction so all terms containing $F_\phi$, $F_\theta$, $\frac{\partial}{\partial \theta}$, $\frac{\partial}{\partial\phi}$ go to zero. Every term contains at least of of these listed elements, therefore
\begin{align*}
\Delta \times F =& 0
\end{align*} 
















\end{document}