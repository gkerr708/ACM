\documentclass[12pt, a4paper]{article}
\usepackage{amsfonts, amssymb,amsmath,float,wrapfig}
\usepackage{graphicx,dsfont,subfiles,hyperref,enumitem,apacite}
\usepackage[margin=1.0in]{geometry}
\linespread{1} 

\title{PHYC 3590 - Advanced Classical Mechanics\\Assignment 6}
\author{Gavin Kerr\\B00801584}
\date{2023-02-28}


\begin{document}

\maketitle

\section{6.16}
From problem 6.1 we have $L = R\int_{\theta_1}^{\theta_2}\sqrt{1+\sin^2\theta\phi'(\theta)^2}d\theta$.
\begin{align*}
\frac{\partial f}{\partial\phi'} =& c
\\
\frac{\partial f}{\partial\phi'} =& \sqrt{1+\sin^2\theta\phi'(\theta)^2}
\\
\frac{\partial f}{\partial\phi'} =& \frac{\sin^2\theta\phi'}{\sqrt{1+\sin^2\theta\phi'(\theta)^2}}
\\
\frac{\sin^2\theta\phi'}{\sqrt{1+\sin^2\theta\phi'(\theta)^2}} =& c
\\
\end{align*}


\section{6.19}
The area of the surface of revolution is defined by the equation. 
\begin{align*}
A =& 2\pi \int_{x_1}^{x_2}y\sqrt{1+x'(y)^2}dy
\\
f =& y\sqrt{1+x'(y)^2}
\\
0 =& {\frac {\partial f}{\partial x}}-{\frac{d }{d y}}{\frac {\partial f}{\partial x'}}
\\
0 =& \frac{d}{dy}\frac{\partial f}{\partial x'}
\\
y_0 =& \frac{\partial f}{\partial x'}
\\
y_0 =& \frac{\partial }{\partial x'}y\sqrt{1+x'^2}
\\
y_0 =& y\frac{x'}{\sqrt{1+x'^2}}
\\
y_0^2(1+x'^2) =& y^2x'^2
\\
y_0^2 =& y^2x'^2 - y_0^2x'^2
\\
y_0^2 =& (y^2 - y_0^2)x'^2
\\
\frac{y_0^2}{y^2 - y_0^2} =& x'^2
\\
x' = \frac{y_0}{\sqrt{y^2 - y_0^2}}
\\
\end{align*}
I am not sure how to simplify this too the desired equation


\pagebreak
\section{6.27}
\begin{align*}
L =& \int ds
\\
ds =& \sqrt{dx^2+dy^2+dz^2}
\\
ds =& \sqrt{x'^2 + y'^2 + z'^2} du
\\
L =& \int \sqrt{x'^2 + y'^2 + z'^2} du
\\ 
f =& \sqrt{x'^2 + y'^2 + z'^2}
\\
\frac{\partial f}{\partial x'} =& \frac{x'}{\sqrt{x'^2 + y'^2 + z'^2}}
\\
\frac{\partial f}{\partial y'} =& \frac{y'}{\sqrt{x'^2 + y'^2 + z'^2}}
\\
\frac{\partial f}{\partial z'} =& \frac{z'}{\sqrt{x'^2 + y'^2 + z'^2}}
\end{align*}
The first order derivative of $f$ with respect to $x$, $y$, and $z$ are all equal to zero therefore
\begin{align*}
\frac{d}{dx}\frac{\partial f}{\partial x'} =& 0
\\
\frac{\partial f}{\partial x'} =& c_1
\\
\frac{d}{dx}\frac{\partial f}{\partial y'} =& 0
\\
\frac{\partial f}{\partial y'} =& c_2
\\
\frac{d}{dx}\frac{\partial f}{\partial z'} =& 0
\\
\frac{\partial f}{\partial z'} =& c_3
\end{align*}
These derivatives are all equal to a constant, therefore $x'$, $y'$ and $z'$ are all equal to constants. This can only be the case if the path is a straight line. 


\pagebreak
\section{7.1}


\section{7.4}


\section{7.8}





\end{document}