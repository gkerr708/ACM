\documentclass[12pt, a4paper]{article}
\usepackage{amsfonts, amssymb,amsmath,float,wrapfig}
\usepackage{graphicx,dsfont,subfiles,hyperref,enumitem,apacite}
\usepackage[margin=1.0in]{geometry}
\linespread{1} 

\title{PHYC 3590 - Advanced Classical Mechanics\\Assignment 6}
\author{Gavin Kerr\\B00801584}
\date{2023-02-28}


\begin{document}

\maketitle

\section{6.16}
From problem 6.1 we have $L = R\int_{\theta_1}^{\theta_2}\sqrt{1+\sin^2\theta\phi'(\theta)^2}d\theta$.
\begin{align*}
\frac{\partial f}{\partial\phi'} =& c
\\
\frac{\partial f}{\partial\phi'} =& \sqrt{1+\sin^2\theta\phi'(\theta)^2}
\\
\frac{\partial f}{\partial\phi'} =& \frac{\sin^2\theta\phi'}{\sqrt{1+\sin^2\theta\phi'(\theta)^2}}
\\
\frac{\sin^2\theta\phi'}{\sqrt{1+\sin^2\theta\phi'(\theta)^2}} =& c
\end{align*}
I don't think I fully understand this but choosing the z axis to pass through point one where $\theta = 0$.


\section{6.19}
The area of the surface of revolution is defined by the equation. 
\begin{align*}
A =& 2\pi \int_{x_1}^{x_2}y\sqrt{1+x'(y)^2}dy
\\
f =& y\sqrt{1+x'(y)^2}
\\
0 =& {\frac {\partial f}{\partial x}}-{\frac{d }{d y}}{\frac {\partial f}{\partial x'}}
\\
0 =& \frac{d}{dy}\frac{\partial f}{\partial x'}
\\
y_0 =& \frac{\partial f}{\partial x'}
\\
y_0 =& \frac{\partial }{\partial x'}y\sqrt{1+x'^2}
\end{align*}
\begin{align*}
y_0 =& y\frac{x'}{\sqrt{1+x'^2}}
\\
y_0^2(1+x'^2) =& y^2x'^2
\\
y_0^2 =& y^2x'^2 - y_0^2x'^2
\\
y_0^2 =& (y^2 - y_0^2)x'^2
\\
\frac{y_0^2}{y^2 - y_0^2} =& x'^2
\\
x' = \frac{y_0}{\sqrt{y^2 - y_0^2}}
\\
\end{align*}
I am not sure how to simplify this too the desired equation.

\section{6.27}
\begin{align*}
L =& \int ds
\\
ds =& \sqrt{dx^2+dy^2+dz^2}
\\
ds =& \sqrt{x'^2 + y'^2 + z'^2} du
\\
L =& \int \sqrt{x'^2 + y'^2 + z'^2} du
\\ 
f =& \sqrt{x'^2 + y'^2 + z'^2}
\\
\frac{\partial f}{\partial x'} =& \frac{x'}{\sqrt{x'^2 + y'^2 + z'^2}}
\\
\frac{\partial f}{\partial y'} =& \frac{y'}{\sqrt{x'^2 + y'^2 + z'^2}}
\\
\frac{\partial f}{\partial z'} =& \frac{z'}{\sqrt{x'^2 + y'^2 + z'^2}}
\end{align*}
The first order derivative of $f$ with respect to $x$, $y$, and $z$ are all equal to zero therefore
\begin{align*}
\frac{d}{dx}\frac{\partial f}{\partial x'} =& 0
\\
\frac{\partial f}{\partial x'} =& c_1
\\
\frac{d}{dx}\frac{\partial f}{\partial y'} =& 0
\\
\frac{\partial f}{\partial y'} =& c_2
\\
\frac{d}{dx}\frac{\partial f}{\partial z'} =& 0
\\
\frac{\partial f}{\partial z'} =& c_3
\end{align*}
These derivatives are all equal to a constant, therefore $x'$, $y'$ and $z'$ are all equal to constants. This can only be the case if the path is a straight line. 


\pagebreak
\section{7.1}
\begin{align*}
T =& \frac{1}{2}m(\dot{x}^2+\dot{y}^2+\dot{z}^2) 
\\
U =& mgz
\\
L =& T - U
\\
L =& \frac{1}{2}m(\dot{x}^2+\dot{y}^2+\dot{z}^2) - mgz
\\
\frac{\partial L}{\partial x} =& \frac{d}{dt}\frac{\partial L}{\partial\dot{x}} = 0
\\
\frac{\partial L}{\partial y} =& \frac{d}{dt}\frac{\partial L}{\partial\dot{y}} = 0
\\
\frac{\partial L}{\partial z} =& \frac{d}{dt}\frac{\partial L}{\partial\dot{z}}
\\
-mg =& \frac{d}{dt}(m\dot{x})
\\
-mg =& m\ddot{x}
\\
-g =& \ddot{x}
\end{align*}
From Newtonian mechanics, you would say that the gravity is the only force acting on the object so the equation would become $F = -mg$, from Newton's second law we have $F = m\ddot{x}$. Combining these two equations, we find the same equation derived above $-mg = m\ddot{x}$.


\section{7.4}
Using Lagrangian mechanics we have
\begin{align*}
U =& mgh = mgy\sin\alpha
\\
T =& \frac{1}{2}m(\dot{x}^2+\dot{y}^2)
\\
L =& \frac{1}{2}m(\dot{x}^2+\dot{y}^2) - mgy\sin\alpha
\\
\frac{\partial L}{\partial x} =& \frac{d}{dt}\frac{\partial L}{\partial\dot{x}}
\\
0 =& \frac{d}{dt}(m\dot{x})
\\
&\boxed{m\ddot{x} = 0}
\\
\frac{\partial L}{\partial y} =& \frac{d}{dt}\frac{\partial L}{\partial\dot{y}}
\\
-mg\sin\alpha =& \frac{d}{dt}(m\dot{y})
\\
&\boxed{m\ddot{y} = -mg\sin\alpha}
\end{align*}
Using Newtonian mechanics we have
\begin{align}
F =& m\ddot{y}
\\
F =& -mg\sin\alpha
\\
m\ddot{y} =& -mg\sin\alpha
\end{align}
The solution using Lagrangian mechanics agrees with the Newtonian solution.



\section{7.8}
\subsection{(a)}
\begin{align*}
U =& \tfrac{1}{2}kx^2 = \tfrac{1}{2}k(x_1-x_2-l)^2
\\
T =& \tfrac{1}{2}m\dot{x_1}^2 + \tfrac{1}{2}m\dot{x_2}^2
\\
L =& \tfrac{1}{2}m\dot{x_1}^2 + \tfrac{1}{2}m\dot{x_2}^2 - \tfrac{1}{2}k(x_1-x_2-l)^2
\end{align*}
\subsection{(b)}
\begin{align*}
X =& \tfrac{1}{2}(x_1 + x_2)
\\
x =& x_1-x_2-l
\\
x_2 =& x_1-x-l
\\
X =& \tfrac{1}{2}(x_1 + x_1-x-l)
\\
x_1 =& \boxed{X + \tfrac{1}{2}(x+l)}
\\
x_1 =& x_2+x+l
\\
X =& \tfrac{1}{2}(x_2 + x + l + x_2)
\\
x_2 =& \boxed{X - \tfrac{1}{2}(x + l)}
\\
L =& 
\tfrac{1}{2}m(\dot{X} + \tfrac{1}{2}\dot{x})^2 + 
\tfrac{1}{2}m(\dot{X} - \tfrac{1}{2}\dot{x})^2 - 
\tfrac{1}{2}k[X - \tfrac{1}{2}(x + l)- X + \tfrac{1}{2}(x + l) - l]^2
\\
L =& 
\tfrac{1}{2}m(\dot{X} + \tfrac{1}{2}\dot{x})^2 + 
\tfrac{1}{2}m(\dot{X} - \tfrac{1}{2}\dot{x})^2 - 
\tfrac{1}{2}kx^2
\\
L =& 
\tfrac{1}{2}m(\dot{X}^2 + \tfrac{1}{2}\dot{x}\dot{X} + \tfrac{1}{2}\dot{x}^2) + 
\tfrac{1}{2}m(\dot{X}^2 - \tfrac{1}{2}\dot{x}\dot{X} + \tfrac{1}{2}\dot{x}^2) - 
\tfrac{1}{2}kx^2
\\
L =& 
\tfrac{1}{2}m(\dot{X}^2 + \tfrac{1}{2}\dot{x}^2) + 
\tfrac{1}{2}m(\dot{X}^2 + \tfrac{1}{2}\dot{x}^2) - 
\tfrac{1}{2}kx^2
\\
L =& \boxed{m\dot{X}^2 + \tfrac{1}{4}m\dot{x}^2 - \tfrac{1}{2}kx^2}
\end{align*}
\begin{align*}
\frac{\partial L}{\partial X} =& \frac{d}{dt}\frac{\partial L}{\partial\dot{X}}
\\
0 =& 2m\ddot{X}
\\
\frac{\partial L}{\partial x} =& \frac{d}{dt}\frac{\partial L}{\partial\dot{x}}
\\
-kx =& \frac{1}{2}m\ddot{x}
\\
\end{align*}
\subsection{(c)}
\begin{align}
0 =& 2m\ddot{X}
\\
\dot{X} =& V
\\
X =& Vt + V_0
\end{align}
This solution says that $X$ is changing with a constant velocity, $V$.
\begin{align*}
0 =& \ddot{x} + \frac{2k}{m}x
\\
&\text{has a general solution of}
\\
x =& A\cos\left(\sqrt{\frac{2k}{m}}t\right) + B\sin\left(\sqrt{\frac{2k}{m}}t\right)
\end{align*}
The solution says that $x$ is oscillating.





















\end{document}