\documentclass[12pt, a4paper]{article}
\usepackage{amsfonts, amssymb,amsmath}
\usepackage{graphicx}
\usepackage{dsfont}
\usepackage[margin=1.0in]{geometry}
\DeclareMathOperator\arctanh{arctanh}
\linespread{1} 

\title{PHYC 3590 - Advanced Classical Mechanics\\Assignment 3}
\author{Gavin Kerr\\B00801584}
\date{2022-01-31}


\begin{document}
\maketitle








\section*{Problem 3.5}
\begin{align}
\vec{p_1} + \vec{p_2} =& \vec{p_1}' + \vec{p_2}'
\\
\vec{v_2} =& \, \vec{p_2} = 0
\\
m_1\vec{v_1} =& m_1\vec{v_1}' + m_2\vec{v_2}' \label{eq1}
\end{align}
\begin{align}
\tfrac{1}{2}m_1v_1^2 + \tfrac{1}{2}m_2v_2^2 =& \tfrac{1}{2}m_1v_1^{'2} + \tfrac{1}{2}m_2v_2^{'2}
\\
v_2 =& 0
\\
m_1v_1^2 =& m_1v_1^{'2} + m_2v_2^{'2} 
\\
m_1 =& m_2
\\
v_1^2 =& v_1^{'2} + v_2^{'2} \label{eq2}
\\
\end{align}
From equation (\ref{eq1}) we have
\begin{align}
m_1\vec{v_1} =& m_1\vec{v_1}' + m_2\vec{v_2}' 
\\
m_1^2\vec{v_1}^2 =& (m_1\vec{v_1}' + m_2\vec{v_2}')^2 
\\
m_1^2\vec{v_1}^2 =& m_1^2 \vec{v_1}'^2  + m_2^2 \vec{v_2}'^2 + 2m_1m_2\vec{v_1}'\cdot\vec{v_2}'
\\
m_1 =& m_2 = m
\\
\vec{v_1}^2 =& \vec{v_1}'^2  + \vec{v_2}'^2 + 2\vec{v_1}'\cdot\vec{v_2}' \label{eq3}
\end{align}
Equating equation (\ref{eq2}) and (\ref{eq3}) we find
\begin{align*}
v_1^{'2} + v_2^{'2} =& \vec{v_1}'^2  + \vec{v_2}'^2 + 2\vec{v_1}'\cdot\vec{v_2}'
\\
\vec{v_1}'\cdot\vec{v_2}' =& 0
\\
|v_1'| \cdot |v_2'| \cos\theta =& 0
\\
\theta =& \boxed{90^o}
\end{align*}








\pagebreak
\section*{Problem 3.11}
\subsection*{(a)}
Show that the equation of motion is $m\dot{v} = -\dot{m}v_{ex} + F^{ext}$.
\begin{align*}
dP =& m \,dv + dm \, v_{ex}
\\
\dot{P} =& \frac{dP}{dt} = F^{ext}
\\
dP =& F^{ext}dt
\\
F^{ext}dt =& m \,dv + dm \, v_{ex}
\\
F^{ext} =& m \frac{dv}{dt} + \frac{dm}{dt} v_{ex}
\\
m\dot{v} =& -\dot{m}v_{ex} + F^{ext}
\\
&\text{QED}
\end{align*}
\subsection*{(b)}
\begin{align*}
m\dot{v} =& -\dot{m}v_{ex} - mg
\\
(m_0-kt)\dot{v} =& kv_{ex} - (m_0-kt)g
\\
(m_0-kt)\frac{dv}{dt} =& kv_{ex} - (m_0-kt)g
\\
dv =& \left(\frac{kv_{ex}}{m_0-kt} - g\right) dt
\\
\int_0^v dv =& \int\left(\frac{kv_{ex}}{m_0-kt} - g\right) dt
\\
v =& \int_0^t \left(\frac{kv_{ex}}{m_0-kt'} - g\right) dt'
\\
v =& \int_0^t \frac{kv_{ex}}{m_0-kt'} dt' - gt
\\
v =& -v_{ex} (\ln|m_0-kt| - \ln|m_0|) - gt
\\
v =&  v_{ex} \ln\left( \frac{m_0}{m_0-kt} \right) - gt
\\
v =&  \boxed{v_{ex} \ln\left( \frac{m_0}{m} \right) - gt}
\end{align*}
\subsection*{(c)}
\begin{align*}
m_0 =& 2\times10^6kg
\\
m_f =& 1\times10^6kg
\\
v_{ex} =& 3000m/s
\\
t =& 120s
\\
v =&  (3000m/s) \ln\left( 
    \frac{2\times10^6kg}{1\times10^6kg} \right) = \boxed{2079m/s}
\\
v =&  (3000m/s) \ln\left( 
    \frac{2\times10^6kg}{1\times10^6kg} \right) - (9.8m/s^2)(120s)
    = \boxed{903m/s}
\end{align*}
Without gravity the rocket would be traveling roughly twice as fast. 
\subsection*{(d)}
If $\dot{m}\,v_{ext} < mg$, the rocket would remain stationary
until the enough fuel mass was ejected so that $\dot{m}\,v_{ext} > mg$. 
Then the rocket would lift off. 






\section*{Problem 3.21}
\begin{align*}
CM =& \frac{1}{m}\int y \, dm 
\\
dm =& \sigma \, dA, \,\,\, y = r\sin\theta
\\
CM =& \frac{1}{m}\int r\sin\theta\sigma dA
\\
\sigma  =& \frac{m}{A} = \frac{m}{1/2\pi R^2} = \frac{2m}{\pi R^2}
\\
dA =& r \, dr  \, d\theta
\\
CM =& \int\int  \frac{2}{\pi R^2} r^2 \, dr \sin\theta  \, d\theta 
\\
CM =& \frac{2}{\pi R^2} \int_0^R r^2  dr  \int_0^{\pi} \sin\theta d\theta 
\\
CM =& \frac{2}{\pi R^2} (R^3/3) (2)
\\
CM =& \boxed{\frac{4R}{3\pi}}
\end{align*}






\section*{Problem 3.27}
\subsection*{(a)}
\begin{align*}
l =& \vec{r} \times \vec{p}
\\
\vec{p} =& m\vec{v}
\\
l =& \vec{r} \times m\vec{v}
\\
\vec{v} =& \frac{d\vec{r}}{dt} = \dot{r}\hat{r} + 
    r\dot{\phi}\hat{\phi}
\\
l =& \vec{r} \times m(\dot{r}\hat{r} + r\dot{\phi}\hat{\phi})
\\
l =& r\hat{r} \times m(\dot{r}\hat{r} + r\dot{\phi}\hat{\phi})
\\
l =& r\hat{r} \times m(\dot{r}\hat{r} + r\dot{\phi}\hat{\phi})
\\
l =& r\hat{r} \times m r\dot{\phi}\hat{\phi}
\\
\omega =& \dot{\phi}
\\
l =& r\hat{r} \times m r\omega\hat{\phi}
\\
l =&  \boxed{ m r^2\omega(\hat{r}\times\hat{\phi}) }
\end{align*}
\subsection*{(b)}
Show that $\frac{dA}{dt} = \tfrac{1}{2}r^2\omega = \frac{l}{2m}$. 
Ler $r$ be the distance between the planet and the sun, let x be the 
distance the planet is traveling perpendicular to r. 
\begin{align*}
A =& \frac{1}{2} r  x
\\
dA =& \frac{1}{2} r \, dx
\\
x =& r\sin\phi
\\
dx =& r\, d\phi
\\
dA =& \frac{1}{2} r \,  r\, d\phi
\\
\frac{dA}{dt} =& \frac{1}{2} r^2 \frac{d\phi}{dt}
\\
\frac{dA}{dt} =& \boxed{\frac{1}{2} r^2 \omega}
\end{align*}
From part (a) we have 
\begin{align}
l =&  m r^2\omega
\\
\frac{l}{2m} =&  \frac{1}{2} r^2\omega
\\
\frac{dA}{dt} =& \frac{1}{2} r^2 \omega
\\
\frac{dA}{dt} =&\boxed{\frac{l}{2m}}
\end{align}






\pagebreak
\section*{Problem 3.32}
\begin{align*}
I =& \sum m_\alpha \rho_\alpha^2
\\
I =& \int_v \textit{Density} \cdot \rho^2 \, dV
\\
dV =& r^2 dr \sin\theta d\theta d\phi
\\
I =& \int_v \textit{Density} \cdot \rho^2 \, r^2 dr \sin\theta d\theta d\phi
\\
\textit{Density} =& \frac{m}{V} = \frac{3m}{4\pi R^3}
\\
I =& \int_v \frac{3m}{4\pi R^3} \rho^2 \, r^2 dr \sin\theta d\theta d\phi
\\
\rho^2 =& r^2\sin^2\theta
\\
I =& \int_v \frac{3m}{4\pi R^3} r^2\sin^2\theta \, r^2 dr \sin\theta d\theta d\phi
\\
I =& \frac{3m}{4\pi R^3}
\int_0^R  r^4 dr 
\int_0^\pi\sin^3\theta d\theta 
\int_0^{2\pi} d\phi
\\
I =& \frac{3m}{4\pi R^3}(R^5/5) (2\pi)
\int_0^\pi\sin^3\theta d\theta
\\
\int_0^\pi\sin^3\theta d\theta =& \int_0^\pi (1-\cos^2\theta)\sin\theta d\theta
\\
=& \int_1^{-1} u^2 - 1 du, \,\, u = \cos\theta, \,\, 
du = - \sin\theta d\theta
\\
=& [u^3/3]_1^{-1} - [u]_1^{-1} = \frac{4}{3}
\\
I =& \frac{3m}{4\pi R^3}(R^5/5) (2\pi) \frac{4}{3}
\\
I =& \frac{2}{5}mR^2
\\
&\text{QED}
\end{align*}
























\end{document}
