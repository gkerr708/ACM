\documentclass[12pt, a4paper]{article}
\usepackage{amsfonts, amssymb,amsmath}
\usepackage{graphicx}
\usepackage{dsfont}
\usepackage[margin=1.0in]{geometry}
\linespread{1} 


\begin{document}

\begin{titlepage}
   \begin{center}
       \vspace*{0.5cm}

       \LARGE{\textbf{Assignment 1}}

       \vspace{1cm}
        \Large{Advanced Classical Mechanics}
            
       \vspace{1cm}
		\small{Gavin Kerr (B00801584)} \\
		\vfill
		\includegraphics[scale=0.65]{dal_logo2.png}
       \vfill
           \large{PHYC xxxx}
            
       \vspace{0.8cm}
     
            
       Physics and Atmospheric Science\\
       Dalhousie University\\
       2022-01-16
            
   \end{center}
\end{titlepage}



\section*{Question 1.16}
\subsection*{(a)}
Show that $|r| = \sqrt{r\cdot r}$. Firstly, let $r = x\hat{x} + y \hat{y} + z \hat{z}$.
\begin{align*}
r =& x\hat{x} + y \hat{y} + z \hat{z}
\\
r\cdot r =& (x\hat{x} + y \hat{y} + z \hat{z})(x\hat{x} + y \hat{y} + z \hat{z})
\\
r\cdot r =& x^2 + y^2 + z^2 = r^2
\\
r\cdot r =& r^2
\\
\sqrt{r\cdot r} =& \sqrt{r^2}
\\
\sqrt{r\cdot r} =& |r|
\\
&\text{QED}
\end{align*}
\subsection*{(b)}
Prove that $r\cdot s$, as defined by (1.7), is the same for any choice of orthogonal axes. 
\begin{align*}
|r|
\end{align*}





\section*{Problem 1.26}
\subsection*{(a)}
For $S$ we have $x=0, y=v_1t$, where $v_1$ is the speed at which the man kicks the puck.
\subsection*{(b)}
For $S'$ relative to $S$ we have $x=v_2t, y=0$, where $v_2$ is the speed of the second observer relative to the first observer ($S$). In the $S'$ frame, the path of the puck would be $x'= x_1-v_2t, y'=y_1 +v_1t$. $(x_1, y_1)$ is the distance between observer 1 \& 2. 
\subsection*{(c)}
For $S''$ relative to $S$, we have $x=\tfrac{1}{2}at^2, y=0$, where $a$ is the acceleration of the third observer. In the $S''$ frame, the puck follows the path $x=x_2-\tfrac{1}{2}at^2 y=y_2+v_1t$. $(x_2,y_2)$ is the distance between the first \& third observer.\\
\\
The $S$ and $S'$ are inertial reference frames. $S''$ is not an inertial reference frame because it is accelerating.





\section*{Problem 1.31}
I think this question needs more. 
\begin{align*}
P_1 + P_2 =& c
\\
\frac{dP_1}{dt} + \frac{dP_2}{dt} =& \frac{dc}{dt}
\\
F_1 + F_2 =& 0
\\
F_1 =& - F_2
\\
\text{QED}
\end{align*}





\pagebreak
\section*{Problem 1.39}
\begin{figure}[h]
  \includegraphics[width=4in]{fig_1.pdf}
  \caption{Diagram}
  \label{fig_1}
\end{figure}
\begin{align*}
g =& -|g|(\sin\phi \hat{x}+\cos\phi \hat{y})
\\
v_0 =& |v_0|(\cos\theta\hat{x}+\sin\theta\hat{y})
\\
x =& |v_0|\cos\theta \,t - \tfrac{1}{2}|g|\sin\phi \,t^2
\\
y =& |v_0|\sin\theta \,t - \tfrac{1}{2}|g|\cos\phi \,t^2
\end{align*}
\begin{align*}
0 =& |v_0|\sin\theta \,t - \tfrac{1}{2}|g|\cos\phi \,t^2
\\
0 =& |v_0|\sin\theta  - \tfrac{1}{2}|g|\cos\phi \,t
\\
\tfrac{1}{2}|g|\cos\phi \,t =& |v_0|\sin\theta
\\
t =& \dfrac{2|v_0|\sin\theta}{|g|\cos\phi}
\end{align*}
This means the the ball touches the ramp at $t = \dfrac{2|v_0|\sin\theta}{|g|\cos\phi}$. Subbing this into the equation for $x$ to find the final position gives us
\begin{align*}
x =& |v_0|\cos\theta \left(\dfrac{2|v_0|\sin\theta}{|g|\cos\phi}\right) - \tfrac{1}{2}|g|\sin\phi \left(\dfrac{2|v_0|\sin\theta}{|g|\cos\phi}\right)^2
\\
x =&  \left(\dfrac{2v_0^2\sin\theta\cos\theta}{|g|\cos\phi}\right) -  \left(\dfrac{2v_0^2\sin^2\theta\sin\phi}{|g|\cos^2\phi}\right)
\\
x =&  \frac{2v_0^2}{|g|}\left(\dfrac{\sin\theta\cos\theta}{\cos\phi} -  \dfrac{\sin^2\theta\sin\phi}{\cos^2\phi}\right)
\\
x =&  \frac{2v_0^2}{|g|}\left(
\dfrac{\sin\theta\cos\theta\cos\phi}{\cos^2\phi} -  
\dfrac{\sin^2\theta\sin\phi}{\cos^2\phi}
\right)
\\
x =&  \frac{2v_0^2}{|g|}\left(
\dfrac{\sin\theta\cos\theta\cos\phi - \sin^2\theta\sin\phi}{\cos^2\phi}
\right)
\\
x =&  \frac{2v_0^2}{|g|}\left(
\dfrac{\sin\theta(\cos\theta\cos\phi - \sin\theta\sin\phi)}{\cos^2\phi}
\right)
\\
x =&  \frac{2v_0^2}{|g|}\left(
\dfrac{\sin\theta\cos(\theta+\phi)}{\cos^2\phi}
\right)
\\
x =& \boxed{R = \dfrac{2v_0^2\sin\theta\cos(\theta+\phi)}{|g|\cos^2\phi}}
\\
&\text{QED}
\end{align*}
To solve for $R_{max}$ we need to find a local max for $R$ as a function of $\theta$.
\begin{align*}
\frac{dR}{d\theta} =& \frac{d}{d\theta}\dfrac{2v_0^2\sin\theta\cos(\theta+\phi)}{|g|\cos^2\phi}
\\
\frac{dR}{d\theta} =& \dfrac{2v_0^2}{|g|\cos^2\phi}\frac{d}{d\theta}\left(\sin\theta\cos(\theta+\phi)\right)
\\
\frac{dR}{d\theta} =& \dfrac{2v_0^2}{|g|\cos^2\phi}\left(
\cos\theta\cos(\theta+\phi) -
\sin\theta\sin(\theta+\phi)
\right)
\\
\frac{dR}{d\theta} =& \dfrac{2v_0^2}{|g|\cos^2\phi}\sin(2\theta+\phi)
\\
0 =& \dfrac{2v_0^2}{|g|\cos^2\phi}\sin(2\theta+\phi)
\\
n\pi =& 2\theta+\phi
\\
\theta_{max} =& \frac{n\pi-\phi}{2}
\end{align*}
I think this solution is wrong. However, next we just sub theta into the equation for $R(\theta_{max})$.





\pagebreak
\section*{Problem 1.46}
\begin{align*}
...
\end{align*}






















\end{document}
